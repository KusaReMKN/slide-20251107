\documentclass[
  lualatex,
  aspectratio=169,
  fleqn,
  14pt,
]{beamer}

\usetheme[progressbar=frametitle]{Metropolis}
%\setbeameroption{show notes on second screen=bottom}

\usepackage{xparse}
\usepackage{mathtools,amssymb}
\usepackage{graphicx,xcolor}
\usepackage{pxrubrica}
\usepackage{calc}
\usepackage[absolute,overlay]{textpos}
%\usepackage{enumitem}
\usepackage[1.7]{bxpdfver}
\usepackage{pdfcomment}
\usepackage{ulem}
\usepackage{stackengine}
\usepackage{appendixnumberbeamer}
\usepackage{tabto}
\usepackage[superscript]{cite}

\RenewDocumentCommand\citeform{m}{[#1]}

% フォント
\usepackage[lining,tabular,sfdefault]{FiraSans}
\usepackage[mathrm=sym,mathbf=sym]{unicode-math}
\setmathfont{Fira Math}
\usepackage[no-math,deluxe,haranoaji]{luatexja-preset}
\RenewDocumentCommand\kanjifamilydefault{}{\gtdefault}

% 打ち消し線関連
\RenewDocumentCommand\ULthickness{}{.1\zh}
% 二重打ち消し
\NewDocumentCommand\dsout{m}{%
  \stackengine{.2\zh}{#1}{%
    \stackengine{-.3\zh}{\sout{\hphantom{#1}}}{\sout{\hphantom{#1}}}{O}{c}{F}{F}{L}%
  }{O}{c}{F}{F}{L}}

% 下に文字置くやつ
\NewDocumentCommand\replace{mm}{%
  \smash{\stackengine{-1\zh}{\dsout{#1}}{#2}{O}{c}{F}{F}{L}}}

% 床関数
\DeclarePairedDelimiter\floor\lfloor\rfloor

\usepackage{hyperref}

\title{郵便を用いた超低速IP通信システムの検討}
\subject{エンジニア作業飲み集会\#XXX}
\author{%
  \ltjsetparameter{autospacing=false, autoxspacing=false}
  上羽 未栞\inst{\dag}\inst{a)}\and
  信濃 眞伊\inst{\dag}\inst{b)}\and
  佐伯 真紘\inst{\dag}\inst{c)}\and
  一式 すみれ\inst{\dag}\inst{d)}}
\institute{
  \inst{\dag}\hspace{.66\zw}東京広域電話網, \url{https://tkytel.github.io/}\\
  a) a.k.a. KusaReMKN, \url{mkn@kusaremkn.com}\\
  b) \url{me@shinanomai.xyz}\\
  c) a.k.a. Nejikugi, \url{me@scrwnl.eu.org}\\
  d) a.k.a. yude, \url{i@yude.jp}
}
\keywords{LoLLIPoP; 郵便; TCP/IP; RFC 1149}
\date{2025-11-07}

\begin{document}

\begin{frame}
  \titlepage
  \thispagestyle{empty}
  \note{
    はじまるよ。

    郵便を用いた超低速IP通信システムの検討と題して、
    東京広域電話網のLoLLIPoPチームを代表して
    上羽未栞が発表するよ。
  }
\end{frame}

\begin{frame}
  \frametitle{今回のおはなし}

  %~\\[-.25\baselineskip]
  \tableofcontents
  \note{
    今回の発表の流れはこんな感じだよ。
    おおよそ25分くらいで進められたらいいな。

    機械式計算器の操作を眺めながら
    現代の電子計算機に通ずる部分があることを紹介していくよ。

    四則演算に加えて平方根も計算できるので、
    百均に売っている電卓と同程度の計算能力を持っていると主張しておくよ。
    機械式計算器もナメたもんじゃないなと思ってもらえると嬉しいな。
  }
\end{frame}

\section*{みかんちゃんについて}
\note{
  まずは自己紹介するよ〜。
}

\begin{frame}
  \frametitle{自称・大天才美少女プログラミング初心者}

  \begin{textblock*}{0.5\paperwidth}(-0.3cm, 3.3cm)
    \includegraphics[width=0.35\paperwidth]{./images/mikanchan.png}
  \end{textblock*}
  \begin{columns}
    \begin{column}{0.30\textwidth}
      \\~\\[-.25\baselineskip]
    \end{column}
    \begin{column}{0.69\textwidth}
      \\~\\[-.25\baselineskip]
      「\ruby{上羽}{うわ|ば} \ruby{未栞}{み|かん}」
      あるいは「\ruby[g]{KusaReMKN}{くされみかん}」\\
      \hspace{1.5\zw}\textbf{みかんちゃん}って呼んでね!
      \\~\\[-.5\baselineskip]

      17{\scriptsize(18)}歳のJK(超重要)\\
      \hspace{1.5\zw}実はプログラマでもエンジニアでもない\\
      \hspace{1.5\zw}古い計算機っぽいものが大好き\\
      \hspace{1.5\zw}自分の得意分野がわからなくなってきた
      \\~\\[-.5\baselineskip]

      Twitterで思想を垂れ流すことが得意\\
      \hspace{1.5\zw}\url{https://kusaremkn.com/}も見てね
    \end{column}
  \end{columns}
  \note{
    自称大天才美少女プログラミング初心者の上羽未栞だよ。
    みかんちゃんって呼ばれると大変喜ぶよ。

    大天才とか偉ぶっているけれど、実はプログラマでもエンジニアでもないよ。
    古い計算機っぽいものが大好きで、いろいろなものに手を出し続けていたら、
    最近は自分の得意分野が何だったのかわからなくなってきたよ。

    Twitterや自分のウェブサイトで思想を垂れ流すのが得意だよ。
    暇な人は眺めてみてね。
  }
\end{frame}

\section{インターネットの通信に思いを馳せる}
\note{

}

\begin{frame}
  \frametitle{インターネットなしでは生きられない!}

  べんりだね
\end{end}

\begin{frame}
  \frametitle{インターネットのしくみ}

  レイヤだね
\end{frame}

\begin{frame}
  \frametitle{情報を伝えるためのしくみ}

  LANケーブル、
  Wi-Fi、
  光ファイバ、
  伝書鳩
\end{frame}

\begin{frame}
  \frametitle{RFC 1149: 1990年4月1日発のジョークRFC}

  鳥類キャリアを用いたIP通信の手法が検討されている\cite{RFC1149}\\
  \hspace{1.5\zw}QoSの提供\cite{RFC2549}やIPv6への対応\cite{RFC6214}など改良・拡張されている


  \note{
    鳥類キャリア、要は伝書鳩を用いて
    IPデータグラムをカプセル化する手法が検討されているよ。
    この手法は1990年に提案されたものだけど、
    その後も改良や拡張が進められていて、
    QoS(Quality of Service)の提供やIPv6への対応なども提案されているよ。
  }
\end{frame}

\appendix

\begin{frame}[standout]
  おわりです

  \note{
    おわりだよ〜。
  }
\end{frame}

\begin{frame}
  \frametitle{参考資料}

  \beamertemplatetextbibitems
  \setbeamerfont{bibliography item}{size=\footnotesize}
  \setbeamerfont{bibliography entry}{size=\footnotesize}
  \setbeamerfont{bibliography entry author}{size=\footnotesize}
  \setbeamerfont{bibliography entry title}{size=\footnotesize}
  \setbeamerfont{bibliography entry location}{size=\footnotesize}
  \setbeamerfont{bibliography entry note}{size=\footnotesize}
  \begin{thebibliography}{1}
    \bibitem{RFC1149}
      Waitzman, D.,
      \newblock
      Standard for the transmission of IP datagrams on avian carriers\textmd,
      \newblock
      \href{https://doi.org/10.17487/RFC1149}{RFC 1149}, April 1990.

    \bibitem{RFC2549}
      Waitzman, D.,
      \newblock
      IP over Avian Carriers with Quality of Service\textmd,
      \newblock
      \href{https://doi.org/10.17487/RFC2549}{RFC 2549}, April 1999.

    \bibitem{RFC6214}
      Carpenter B., Hinden  R.,
      \newblock
      Adaptation of RFC 1149 for IPv6\textmd,
      \newblock
      \href{https://doi.org/10.17487/RFC6214}{RFC 6214}, April 2011.
  \end{thebibliography}
\end{frame}

\begin{frame}
  \frametitle{このスライドについて}

  Written in November 2025.

  Permanent ID of this document: \texttt{2976cf5d5f923407}.

  Copyright © 2025 KusaReMKN.

  特記無き場合、プログラムやソースコードは MIT License で、\\
  \hspace{1.5\zw}それ以外のコンテンツは CC-BY 4.0 で利用可能です。\\
  \hspace{1.5\zw}一部の画像には別のライセンスが適用されるかもしれません。
\end{frame}

\end{document}
% ex: se et ts=2 :
